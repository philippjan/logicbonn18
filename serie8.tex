\documentclass{article}
\usepackage[utf8]{inputenc}
\usepackage[T1]{fontenc}
\usepackage{lmodern}
\usepackage[ngerman]{babel}
\usepackage{amsmath,amssymb,amsfonts,amsthm,mathtools,sansmath}
\usepackage{cleveref}

%Commands
\newcommand{\model}{\mathfrak{M}}
\newcommand{\claim}{\textsf{Behauptung}:\hspace{0,2cm}}
\makeatletter
\def\fall#1{\forall #1\@ifnextchar\bgroup{\,\fall}{:\,}}
\makeatother

%Umgebungen
\newtheorem{thm}{Theorem}[section] 
\theoremstyle{definition}
\newtheorem{defn}[thm]{Definition}
\theoremstyle{plain}
\newtheorem{cor}[thm]{Korollar}
\newtheorem{lem}[thm]{Lemma}
\newtheorem{propo}[thm]{Proposition}
\newtheorem{axiom}[thm]{Axiom}
\theoremstyle{remark}
\newtheorem{remark}[thm]{Remark}
\newtheorem{example}[thm]{Example}

%Aufgaben-Command
\newcommand{\aufgabe}[1]{
	{
		\vspace*{0.5cm}
		\textsf{\textbf{Aufgabe #1}}
		\vspace*{0.2cm}

	}
}
%unteraufgabe
\newcommand{\unteraufgabe}[1]{
	{
		\vspace*{0.2cm}
\noindent\textsf{(#1)}
}
}
%Teilaufgabe
\newcommand{\teilaufgabe}[1]{
	{
\noindent 
\vspace*{0.2cm}
\hspace*{0,1 cm}
\textsf{#1)}
}
}

\newcommand{\induktionsanfang}{
	{
		\vspace*{0.1cm}
		\noindent
		\textsf{Induktionsanfang:}
	}
}

\newcommand{\induktionsschritt}{
	{
		\vspace*{0.1cm}
		\noindent
		\textsf{Induktionsschritt:}
	}
}

\title{Serie 8}
\author{Philipp Stassen, Felix Jäger, Lisa Krebber}
\begin{document}
\maketitle
\aufgabe2 Sei $S$ eine Sprache und $\Phi$ eine konsistente Menge von universellen $S$-Sätzen, in denen das symbol $\equiv$ nicht auftaucht.

\claim Es gibt ein Model $\model$ von $\Phi$ mit $\model(t_1) \neq \model(t_2)$ für alle paarweise verschiedenen Terme $t_0$ und $t_1$.
\begin{proof}
	Wir konstruieren ein Termmodell $\mathfrak{T}^{\Phi}$ nach dem Vorbild der Vorlesung (\textit{Definition} 52). Die Relation $\sim$ sei trivial und identifiziere bloß syntaktisch identische Terme. Damit ist $T^S$ die Grundmenge von $\mathfrak{T}^{\Phi}$. Damit folgt die gewünschte Eigenschaft, dass $\mathfrak{T}^{\Phi}(t_0)\neq\mathfrak{T}^{\Phi}(t_1)$ für paarweise verschiedene Terme $t_0$ und $t_1$ direkt aus der Definition von $\mathfrak{T}^{\Phi}$. \medskip

	(1) Es bleibt zu zeigen, dass $\mathfrak{T}^{\Phi}\vDash \Phi$. Wir wissen, dass $\mathfrak{T}^{\Phi}$ alle atomaren Formeln aus $\Phi$ erfüllt (Theorem 54). Um die Aussage $\mathfrak{T}^{\Phi}\vDash \Phi$ zu zeigen, müssen wir ähnlich wie bei der Konstruktion von Henkin Mengen vorgehen. Da $\Phi$ nach Annahme konsistent ist, und die Ableitungs Vollständigkeit nach \textit{Theorem} 61 ohne Komplikationen auch für diesen Fall folgt, genügt es zu zeigen, dass $\Phi$ ''Zeugen'' enthält. Wir schwächen die Aussage etwas ab, und verlangen keine Zeugen. Im Gegenzug müssen wir die Sprache nicht erweitern, da $\Phi$ diese schwächere Annahme bereits erfüllt.\medskip

	(2) \claim Für $\varphi = \forall x \psi \in \Phi$ gibt es Terme $t_0,...,t_{n-1}$ sodass $\Phi \vdash \neg\varphi \to \neg(\bigwedge_{0\leq i< n}\psi\frac{t_i}{x})$.

	Beweis.
	Die Behauptung folgt aus Herbrands Theorem und dem Vollständigkeitssatz. \textit{qed}(2) \medskip

	(3) Nun müssen wir um (1) zu erhalten noch zeigen, dass \emph{Lemma 56 c)} und \emph{Theorem 57} in leicht abgewandelter Form trotzdem gelten.
	
	\emph{Lemma 56 c)} Für alle $t_0,...,t_{n-1}\in T^S$ gilt $\Phi \vdash \bigwedge_{0\leq i <n} \psi \frac{t_i}{x}$ gdw $\Phi \vdash \forall x \,\psi$. 

	Der Beweis wiederholt die Argumentation des originalen Lemmas.\medskip

	\emph{Beweis Theorem 57} Lediglich Part $iv)$ des Beweises, also der Induktionsschrit zum Allquantor, muss modifiziert werden.
\end{proof}
\end{document}
