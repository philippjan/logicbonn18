\documentclass{article}
\usepackage[utf8]{inputenc}
\usepackage[T1]{fontenc}
\usepackage{lmodern}
\usepackage[ngerman]{babel}
\usepackage{amsmath,amssymb,amsfonts,amsthm,mathtools,sansmath}
\usepackage{cleveref}

%Commands
\newcommand{\model}{\mathfrak{M}}
\newcommand{\claim}{\textsf{Behauptung}:\hspace{0,2cm}}
\makeatletter
\def\fall#1{\forall #1\@ifnextchar\bgroup{\,\fall}{:\,}}
\def\exis#1{\exists #1\@ifnextchar\bgroup{\,\exis}{:\,}}
\makeatother

%Umgebungen
\newtheorem{thm}{Theorem}[section] 
\theoremstyle{definition}
\newtheorem{defn}[thm]{Definition}
\theoremstyle{plain}
\newtheorem{cor}[thm]{Korollar}
\newtheorem{lem}[thm]{Lemma}
\newtheorem{propo}[thm]{Proposition}
\newtheorem{axiom}[thm]{Axiom}
\theoremstyle{remark}
\newtheorem{remark}[thm]{Remark}
\newtheorem{example}[thm]{Example}

%Aufgaben-Command
\newcommand{\aufgabe}[1]{
	{
		\vspace*{0.5cm}
		\textsf{\textbf{Aufgabe #1}}
		\vspace*{0.2cm}

	}
}
%unteraufgabe
\newcommand{\unteraufgabe}[1]{
	{
		\vspace*{0.2cm}
\noindent\textsf{(#1)}
}
}
%Teilaufgabe
\newcommand{\teilaufgabe}[1]{
	{
\noindent 
\vspace*{0.2cm}
\hspace*{0,1 cm}
\textsf{#1)}
}
}

\newcommand{\induktionsanfang}{
	{
		\vspace*{0.1cm}
		\noindent
		\textsf{Induktionsanfang:}
	}
}

\newcommand{\induktionsschritt}{
	{
		\vspace*{0.1cm}
		\noindent
		\textsf{Induktionsschritt:}
	}
}

\title{Serie 5}
\author{Philipp Stassen, Felix Jäger, Lisa Krebber}
\begin{document}
\maketitle
\aufgabe{15}
\unteraufgabe{1} Sei $S$ eine Sprache und $\Phi$ eine Menge von $S$-Sätzen, die endliche Modelle besitzt. \claim Die Klasse aller $S$-Strukturen $\mathfrak{A}$ mit $\mathfrak{A}\vDash\Phi$ ist genau dann axiomatisierbar, wenn eine $N\in \mathbb{N}$ existiert, sodass die Trägermenge jeder $S$-Struktur $\mathfrak{A}$ mit $\mathfrak{A}\vDash \Phi$ höchstens $N$ Elemente enthält.
\begin{proof}
	''$\Longleftarrow$'' Man fügt $\Phi$ einen Satz hinzu, der ausdrückt, dass es höchstens $N$ verschiedene Variablen gibt. Es lässt sich formulieren, dass es nur $n$ verschiedene Variablen gibt:
	\begin{align}
		\exis{v_1}{v_2...}{v_n}&\bigwedge_{\substack{i=1,j=1 \\ i< j}}^n\neg(v_i\equiv v_j) \\
		\fall{x} &\bigvee_{i=1}^n x\equiv v_i
	\end{align}
Fügt man diese Sätze für alle $n\leq N$ nun per Disjunktion zusammen, erhält man einen langen Satz, der vereint mit $\Phi$ die Modell Klasse axiomatisiert.  \smallskip

''$\Longrightarrow$'' Angenommen kein solches $N$ existiert und die Klasse der Modelle, die $\Phi$ erfüllen und endlich sind, ist axiomatiersierbar durch $\Phi'$, dann gibt es zu jedem $n\in \mathbb{N}$ Modelle $\mathcal{M}_n\vDash \Phi'$ mit $|\mathcal{M}(\forall)|=n$. Aus \emph{Theorem 71} folgt nun, dass $\Phi'$ auch ein unendliches Modell besitzt. Dies widerspricht der Annahme.
\end{proof}
\unteraufgabe{2} Es sei $\varphi$ ein $S_{Gr}$-Satz und $\Phi_{Gr}$ die üblichen Gruppen Axiome. \claim Gilt $\varphi$ für jede unendliche Gruppe, dann existiert ein $N$, sodass $\varphi$ für jede Gruppe mit mindestens $N$ Elementen gilt.
\begin{proof}
	Wir definieren
	\begin{align}
		\Phi'=\Phi \cup \bigcup_{n\in\mathbb{N}}\{\psi_{\geq n}\} \text{ mit} \\
		\psi_{\geq n} = \exis{v_1...}{v_n}\bigwedge_{i<j<n}\neg(v_i\equiv v_j).
	\end{align}
Da $\varphi$ für jede unendliche Gruppe gilt, d.h. $\mathcal{M}\vDash\varphi$ für alle $\mathcal{M}\in \mathrm{Mod}_{S_{Gr}}(\Phi')$, folgt, dass $\Phi'\vDash \varphi$. Nach dem Kompaktheitssatz existiert nun eine endliche Teilmenge $\Phi_0\subset \Phi'$, sodass $\Phi_0\vDash\varphi$. Wähle $n_0$ sodass $\Phi_0\subseteq \Phi'\cup \bigcup_{n<n_0}\{\psi_{\geq n}\}$. Nun ist jedes Modell $\mathcal{M}\in\mathrm{Mod}_{S_{Gr}}(\Phi_0)$ ein Modell von $\varphi$ und eine Gruppe mit mindestens $n_0:=N$ Elementen.
\end{proof}
\aufgabe{17} Es seien $\Phi$ die Körperaxiome, $K_0$ ein Körper und $S_{K_0}$ die erweiterte Sprache aus \emph{Aufgabe 16}. Wir definieren
\begin{align}
	\Phi_{0}:= \Phi\cup\bigcup_{n>0}\{\fall{a_1...}{a_n}\exists x\,:\sum_{i=0}^{n}a_ix^i\equiv 0\}.
\end{align}
Der Satz von Kronecker sagt nun, dass jede endliche Teilmenge $\Phi_0\subseteq \Phi'$ erfüllbar ist, da wir in diesem Fall ein Modell $\mathfrak{A}$ haben, sodass $K_0\hookrightarrow \mathfrak{A}$ und die (endlich vielen) Polynome dort eine Nullstelle besitzen.

Nun folgt aus dem Kompaktheitssatz, dass $\Phi'$ ebenfalls erfüllbar ist. Sei $K_1$ so, dass $K_0\hookrightarrow K_1$ und $K_1\vDash \Phi'$. Wir iterieren dieses Verfahren, indem wir nun die Sprache $S_{K_{i+1}}$ und die Formelmenge
\begin{align}
	\Phi_{i+1}:=\Phi_{i}\cup\bigcup_{n>0}\{\fall{a_1...}{a_n}\exists x\,:\sum_{i=0}^{n}a_ix^i\equiv 0\}
\end{align}
(evtl. muss man die Variablen geschickter indizieren...) betrachten. 
Daraus erhalten wir eine Kette $\langle K_n|n\in\mathbb{N}\rangle$, sodass $K_n\hookrightarrow K_{n+1}$ und $K_{n+1}\vDash \Phi_{n}$. \smallskip

Wir können nun $K:=\bigcup_{n\in\mathbb{N}}K_n$ definieren. Hier ist $K_n\subset K_{n+1}$ als Teilmenge aufgefasst, indem $K_n$ mit seinem Bild unter der Einbettung $K_n\hookrightarrow K_{n+1}$ identifiziert wird. (Vielleicht könnte man auch stattdessen das Tensorprodukt der Folge betrachten und spart sich dann diese Schummelei...). $K$ ist nun ein algebraisch abgeschlossener Körper und $K_0\hookrightarrow K$.
%Nach Konstruktion ist $\forall i\, :\, \mathrm{Mod}_{S_{K_i}}(\Phi_{i})\neq \emptyset$. 
%\begin{align}
%	\mathrm{Mod}_{S_{K_i}}(\Phi_{i})&\hookrightarrow\mathrm{Mod}_{S_{K_{i+1}}}(\Phi_{i+1}) \text{ mit} \\
%	\mathfrak{A} &\mapsto \mathfrak{A}_*
%\end{align}
\end{document}

